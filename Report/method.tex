%%% lorem.tex --- 
%% 
%% Filename: lorem.tex
%% Description: 
%% Author: Ola Leifler
%% Maintainer: 
%% Created: Wed Nov 10 09:59:23 2010 (CET)
%% Version: $Id$
%% Version: 
%% Last-Updated: Wed Nov 10 09:59:47 2010 (CET)
%%           By: Ola Leifler
%%     Update #: 2
%% URL: 
%% Keywords: 
%% Compatibility: 
%% 
%%%%%%%%%%%%%%%%%%%%%%%%%%%%%%%%%%%%%%%%%%%%%%%%%%%%%%%%%%%%%%%%%%%%%%
%% 
%%% Commentary: 
%% 
%% 
%% 
%%%%%%%%%%%%%%%%%%%%%%%%%%%%%%%%%%%%%%%%%%%%%%%%%%%%%%%%%%%%%%%%%%%%%%
%% 
%%% Change log:
%% 
%% 
%% RCS $Log$
%%%%%%%%%%%%%%%%%%%%%%%%%%%%%%%%%%%%%%%%%%%%%%%%%%%%%%%%%%%%%%%%%%%%%%
%% 
%%% Code:

\chapter{Methods}
\label{cha:method}

\section{Preprocessing}
The data for this project was delivered by Östergötatrafiken. The whole dataset is over 300GB in size. To be able to feed the data into our algorithms preprocessing had to be done. 

\subsection{Data structure}
The provided data consists of one file per day over 90 day. Each day has an approximate size of 5GB. Within the data there are over 20 event-types that represent the state of a bus during a day. For our project we have used four types of events and discarded data with other events. The four events are:\\
\begin{description}
\item[ObservedPositionEvent:] Gets triggered every second, contains the GPS data of a given bus.
\item[EnteredEvent:] Gets triggered when the bus is within a certain distance to a bus-station. Is used to split the journey into segments.
\item[JourneyStartedEvent:] Gets triggered when the bus is assigned a new journey. Is used to determine which line a bus is currently serving.
\item[JourneyCompletedEvent:] Gets triggered when the bus has completed a journey. Is used as a flag to determine when a journey has ended.
\end{description}

\subsection{Loading data}
We have used jupyter notebooks as the main way to develop our code to pre-process. Data operations such as filtering are done with the python pandas package.
%%%%%%%%%%%%%%%%%%%%%%%%%%%%%%%%%%%%%%%%%%%%%%%%%%%%%%%%%%%%%%%%%%%%%%
%%% lorem.tex ends here

%%% Local Variables: 
%%% mode: latex
%%% TeX-master: "demothesis"
%%% End: 
