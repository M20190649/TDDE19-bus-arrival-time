%%% lorem.tex --- 
%% 
%% Filename: lorem.tex
%% Description: 
%% Author: Ola Leifler
%% Maintainer: 
%% Created: Wed Nov 10 09:59:23 2010 (CET)
%% Version: $Id$
%% Version: 
%% Last-Updated: Wed Nov 10 09:59:47 2010 (CET)
%%           By: Ola Leifler
%%     Update #: 2
%% URL: 
%% Keywords: 
%% Compatibility: 
%% 
%%%%%%%%%%%%%%%%%%%%%%%%%%%%%%%%%%%%%%%%%%%%%%%%%%%%%%%%%%%%%%%%%%%%%%
%% 
%%% Commentary: 
%% 
%% 
%% 
%%%%%%%%%%%%%%%%%%%%%%%%%%%%%%%%%%%%%%%%%%%%%%%%%%%%%%%%%%%%%%%%%%%%%%
%% 
%%% Change log:
%% 
%% 
%% RCS $Log$
%%%%%%%%%%%%%%%%%%%%%%%%%%%%%%%%%%%%%%%%%%%%%%%%%%%%%%%%%%%%%%%%%%%%%%
%% 
%%% Code:

\chapter{Discussion}
\label{cha:discussion}

This chapter starts out with a discussion about the results of the project, then the method and lastly some future work.

\section{Results}
\label{sec:discussion-results} 
Since the only complete results are for bus line 3, the discussion will be limited to those. 

There are differences in the data sets due to the division of the 20\%, 40\%, 60\% and 80\% sections being done after stop compression. This results in that the 20\% division in a stop compressed trajectory is further in to the journey than in a trajectory that is not stop compressed. If the data was split on time stamp that would not differ, but, since the data is split on row number, it does. This, most likely, has an effect on the results and therefore models trained on stop compressed data cannot really be compared to models trained on data that was not stop compressed.

As can be seen in table \ref{tbl:models-mae-and-mape-203}, the neural networks performed best in terms of both MAE and MAPE, on both the regular data and the stop compressed data. The baseline method and M1 seems to be on par, and their MAPE blows up when predicting on smaller parts of the segment, for the 80\% prediction, the prediction is bigger than the time left, yielding an MAPE above 100\%. The GPR however preforms quite poorly in comparison to the neural network models.

When considering the evaluation segment wise, see table \ref{tbl:model-mae-of-segs-203}, the neural network approaches also dominates, the stop compressed models yields the best results in terms of MAE, specially M2 and M3. Table \ref{fig:model-mape-of-segs-203} shows that in relative terms, the non-compressed models sometimes yields better results. This is probably due to the differences in the trajectory splits described earlier. However, M2 and M3 still outperforms the other models.

In terms of complexity, the neural networks require much less resources than the GPR in order to make a prediction with the current implementation, and the GPR will become even more expensive given more trajectories to compare with. The GPR would however, most likely, improve its performance given more trajectories.

\section{Method}
\label{sec:discussion-method}
\subsection{Evaluation Methods}
There a two main differences in the methods that we have used. The baseline method and the M1 model do predictions on complete segments. Those two methods do not have the capability of making sophisticated predictions within a segment; therefore measurements inter-segment predictions are made by subtracting the time spent on the segment from the time predicted for the whole segment. This leads to an increased MAPE as the prediction error remains constant, but with using less and less distance to predict on (80,60,40,20), the constant error corresponds to larger parts of the sub-segment.

\subsection{Neural Networks}
There were eight evaluated models in total. Some of them and their characteristics will be discussed below. The neural networks that were trained and evaluated on stop compressed data performed much better overall with regards to both MAE and MAPE. This was expected since the stop compression removes most of the data points observed during dwell time when the bus is standing still which is the main source of error for the models using non-compressed data. Models M2,M3 and M4 all have similar performance overall whereas the performance of M1 is on par with the baseline model. The reason for the poor performance of M1 is that, as can be seen in figure \ref{fig:ann-m1}, the prediction error of model M1 will be static over each segment due to the model predicting travel times for entire segments rather than time left until next bus stop like the other models.

For the models using non-compressed data, M4 performed slightly better than M3 and M2. A likely reason is that the RNN managed to model the dwell time better than the other models, although dwell time is still the main cause of error for this model. For the models using compressed data, M2 and M3 had similar results and both outperformed M4. Due to hardware limitations the longest sequence that could be used as input for the RNN in M4 was 20 data points. It would be interesting to know if longer sequences would yield better results.

The neural networks seem to be able to generalize to other bus lines fairly well without having to change network parameters. Comparing the MAPE for bus line 3 and 11 in tables \ref{tbl:models-mae-and-mape-203} and \ref{tbl:models-mae-and-mape-211} it can be seen that it is a bit higher for bus 11, although this was expected since that bus travels through areas with heavy traffic which would likely be hard to model.

\subsection{Kalman filtering}
Some of the cited sources used a Kalman filter on top of the predictions with great prediction improvement. A filter was applied to all NN models as well but without success. This was due to the models constantly overpredicting or underpredicting the time left, meaning that the predictions were not distributed around the true value. This left the Kalman filter unable to correct the error since the implementation was of the most basic type, only able to handle gaussian noise. 

There are models designed for non-gaussian noise that could be implemented for future work.

\subsection{Gaussian Process Regression}
The Gaussian Process regression approach only used 200 trajectories to train on since it was computationally expensive to compare data with each trajectory. Using fewer comparison trajectories means that the model considers fewer kinds of behaviours which makes it worse than a model considering all comparison trajectories.

Apart from the fact that not all training data was used, the models performance was worse than expected. At times it was even as bad as the baseline model. Furthermore, it would be expected that the MAE decrease as larger parts of trajectories are observed by the model, since it can find likely trajectories easier with more data points, but this was not the case. This implies a bug in the implementation or, more likely, that the synchronisation GP is not choosen well. The synchronisation GP is by far the most critical and the most difficult piece of the model, and if it does not synchronise trajectories properly the likelihood computed to weight predictions will be nonsensical.

\subsection{Stop compression}
For some models, the data was pre-processed with stop compression, as described in section~\ref{sec:stop-compression}. The 20\%, 40\%, 60\% and 80\% sections were extracted after pre-processing, which means that the stop compressed and non-stop compressed data are not divided into the same sections. I.e. the 20\% prediction on one data set may not be on the same timestamp as in the other data set.

\section{Future Work}
\label{sec:future-work}
This section describes areas of improvements worth investigating for the different models.

\subsection{Neural Network}
The main source of errors in the neural network models come from dwell time. If you could introduce knowledge about the behaviour of dwell time, like the actual bus schedule, better predictions could be achieved. The recurrent neural network was an attempt at modelling dwell times but they still caused errors. Using longer sequences could possibly improve predictions.

Having the schedule of the buses available would allow for some interesting new approaches. At any given station the current delay could be fed into the neural network. Since bus driver drives faster and dwells less when they are late, having this as an input could improve prediction accuracy. Furthermore, a very simple baseline model could be created using the schedule. By simply using the offset of the last bus it should be possible to make acceptable predictions, as two subsequent bus often have similar delays due to traffic conditions.

The recurrent neural network model (NN - M4) could perform better if a longer sequence of data points were used. Due to hardware constraints, the model in this work uses sequences of 20 data points as input, more than that required additional RAM. Doing this would either require more capable hardware or some exploration of how to split the training up into multiple sessions where the data is divided into smaller chunks.

A different approach to making predictions with neural networks could be to output distributions instead of single point estimates. This would yield more information about the uncertainty of the results.

\subsection{Gaussian Process Regression}
The most central part of the model is the synchronisation of trajectories, which is consequently the most important one to get right. For this implementation, the trajectory to train the synchronisation GP on was hand picked by manually trying different ones and observing plots. The trajectory used to train the synchronisation GP made a big difference on how good the GP mapped coordinates into progress, and it is very likely that the one picked is not the best since no methodical approach was taken. Trying something more methodical for doing this would be a great improvement, and something as simple as taking the trajectory with mean arrival time would be something to start with. To make the synchronisation GP better support data was generated in each observed data point, but it could very well be done on a more fine-grained interval, which may improve its performance further.

When the model makes predictions it weights previously observed trajectories using their likelihood. It is entirely possible that the likelihood can be weighted with a parameter learned from data to achieve a better performance in the final predictions.

%%%%%%%%%%%%%%%%%%%%%%%%%%%%%%%%%%%%%%%%%%%%%%%%%%%%%%%%%%%%%%%%%%%%%%
%%% lorem.tex ends here

%%% Local Variables: 
%%% mode: latex
%%% TeX-master: "demothesis"
%%% End: 
