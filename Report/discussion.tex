%%% lorem.tex --- 
%% 
%% Filename: lorem.tex
%% Description: 
%% Author: Ola Leifler
%% Maintainer: 
%% Created: Wed Nov 10 09:59:23 2010 (CET)
%% Version: $Id$
%% Version: 
%% Last-Updated: Wed Nov 10 09:59:47 2010 (CET)
%%           By: Ola Leifler
%%     Update #: 2
%% URL: 
%% Keywords: 
%% Compatibility: 
%% 
%%%%%%%%%%%%%%%%%%%%%%%%%%%%%%%%%%%%%%%%%%%%%%%%%%%%%%%%%%%%%%%%%%%%%%
%% 
%%% Commentary: 
%% 
%% 
%% 
%%%%%%%%%%%%%%%%%%%%%%%%%%%%%%%%%%%%%%%%%%%%%%%%%%%%%%%%%%%%%%%%%%%%%%
%% 
%%% Change log:
%% 
%% 
%% RCS $Log$
%%%%%%%%%%%%%%%%%%%%%%%%%%%%%%%%%%%%%%%%%%%%%%%%%%%%%%%%%%%%%%%%%%%%%%
%% 
%%% Code:

\chapter{Discussion}
\label{cha:discussion}

This chapter contains the following sub-headings.

\section{Results}
\label{sec:discussion-results}


\section{Method}
\label{sec:discussion-method}
\subsection{Evaluation Methods}
There a two main differences in the methods that we have used. The baseline method and the M1 model do predictions on complete segments. Those two methods do not have the capability of making sophisticated predictions within a segment, therefore measurements inter-segment predictions are done by subtracting the time spent on the segment from the time predicted for the whole segment. This leads to an increased MAPE as the prediction error remains constant but with using less and less distance to predict on (80,60,40,20), the constant error corresponds to larger parts of the sub-segment.

\subsection{Gaussian Process regression}
The Gaussian Process regression approach did not use the full training set since it was computationally expensive to compare data with each trajectory. Using fewer comparison trajectories means that the model does consider fewer kinds of behaviours and may, therefore, have worse predictions with a model considering all comparison trajectories, however, it will be faster.

The synchronisation was challenging to get to work well. The mapping between spatial coordinates and progress was done with a synchronisation GP trained on a hand-picked trajectory that seemed to represent the whole dataset reasonably well, however changing that trajectory, made a big difference on how good the GP mapped coordinates into progress. Some effort was put into trying out different trajectories to train the synchronisation GP, but finding a better one could improve the synchronisation process.

Another issue was the support data. It greatly improved the synchronisation as seen in figures~\ref{fig:heightmap-without-support} and~\ref{fig:heightmap-with-support}. This process could also be refined to make the contour lines even more orthogonal to the progression and to be equally spaced along the trajectory. 

The final part of the approach was the prediction, where the weight was put on each trajectory given by how well it explained the observed trajectory. This is another part that could be explored further, to see if, for example, it is better performance wise to punish trajectories with low weights even more.

\section{Future Work}
\label{sec:future-work}
Having the time schedule of the buses available would allow for some interesting new approaches. At any given station the current delay could be fed into the neural network. Since bus driver drive faster and dwell less when they are late, having this as an input could improve prediction accuracy. Furthermore a very simple baseline model could be created using the time schedule. By simple using the offset of the last bus it should be possible to make acceptable predictions, as two subsequent bus often have similar delays due to traffic conditions.

The recurrent neural network model (ANN model 4) could possibly perform better if a longer sequence of data points were used. Due to hardware constraints the model in this work uses sequences of 20 data points as input, more than that required additional RAM. Doing this would either require more capable hardware or some exploration of how to split the training up into multiple sessions where the data is divided into smaller chunks.

%%%%%%%%%%%%%%%%%%%%%%%%%%%%%%%%%%%%%%%%%%%%%%%%%%%%%%%%%%%%%%%%%%%%%%
%%% lorem.tex ends here

%%% Local Variables: 
%%% mode: latex
%%% TeX-master: "demothesis"
%%% End: 
