%%% lorem.tex --- 
%% 
%% Filename: lorem.tex
%% Description: 
%% Author: Ola Leifler
%% Maintainer: 
%% Created: Wed Nov 10 09:59:23 2010 (CET)
%% Version: $Id$
%% Version: 
%% Last-Updated: Tue Oct  4 11:58:17 2016 (+0200)
%%           By: Ola Leifler
%%     Update #: 7
%% URL: 
%% Keywords: 
%% Compatibility: 
%% 
%%%%%%%%%%%%%%%%%%%%%%%%%%%%%%%%%%%%%%%%%%%%%%%%%%%%%%%%%%%%%%%%%%%%%%
%% 
%%% Commentary: 
%% 
%% 
%% 
%%%%%%%%%%%%%%%%%%%%%%%%%%%%%%%%%%%%%%%%%%%%%%%%%%%%%%%%%%%%%%%%%%%%%%
%% 
%%% Change log:
%% 
%% 
%% RCS $Log$
%%%%%%%%%%%%%%%%%%%%%%%%%%%%%%%%%%%%%%%%%%%%%%%%%%%%%%%%%%%%%%%%%%%%%%
%% 
%%% Code:
\chapter{Theory}
\label{cha:theory}
This chapter aims to explain concepts and methods that are used is our implementations.

\section{Gaussion Processes}
A Gaussian Process (\textit{GP}) is, in statistical terms, a stochastic process such that any set of the variables have a multivariate normal distribution. Such a distribution is defined by a mean vector $\mu$ and covariance matrix $\sigma$, where every random variable spans one dimension. Viewed through the lens of machine learning, the GP serves as a non-parametric regression model.

\subsection{Gaussian Process Regression}
GP regression (also known as kriging) is a non-parametric generative model that builds on the assumption that all data points are drawn from a multivariate normal distribution. A GP $f$ is consequently defined as $f \sim \mathcal{N}(0, \Sigma)$. Zero mean can be assumed without loss of generality and is done for mathematical convenience, but in practice $\mu$ could be any suitable mean vector. The covariance matrix $\Sigma$ is thus the only free parameter to be chosen. Unfortunately, a covariance matrix over a continuous function would exist only in an infinite dimensional space, which is indeed problematic. However, the spaces we are interested in have an inner product corresponding to some covariance function $c(x, x') = cov(f(x), f(x'))$ for the process $f$ and data points $x$, $x'$, and this function can be kernelised as $k(\theta, x, x') = cov(f(x), f(x'))$ for some valid kernel function $k$ and its hyperparameters $\theta$. It is consequently possible to compute $\Sigma$ given a set of data points and a kernel function, circumventing the need to construct infinite dimensional spaces.

Thanks to the kernel trick the choice of parameters is consequently not a covariance matrix $\Sigma$ but a kernel $k(\theta, x, x')$ and its hyperparameters $\theta$. This choice represent our prior over $f$, and in particular it represents how smooth we believe $f$ is by imposing certain covariance on $f(x)$ for nearby points. This is the key idea of a GP regression and what makes it very flexible: It allows us to specify a prior over $f$ as a function of $x$, without even knowing $f$.

\section{Artificial Neural Networks}
Artificial neural networks have shown to to be useful when predicting travel times due to their ability to model nonlinear relationships between features \cite{brazilANN}\cite{malaysiaANN}. This section introduces the parameters that have been considered when creating artificial neural networks for this report.

\subsection{Activation Functions}
The purpose of the activation function is to compute the hidden layer values \cite{Goodfellow-et-al-2016}. Two of the most popular activation functions are the sigmoid function:

\begin{equation} 
	f(x) = \frac{1}{1+e^{-x}} 
\end{equation}

and the rectifier function:

\begin{equation} 
	f(x) = max\{0,x\}
\end{equation}

Using the rectifier function leads to less computationally complex learning than when using the sigmoid function although problems can occur where the backpropagation is blocked by a ``dead'' neuron due to the hard saturation at 0 \cite{pmlr-v15-glorot11a}.

 \subsection{Loss Functions}
For an artificial neural network to be able to update its weights it needs a loss function that should be minimized in the case of gradient descent \cite{Goodfellow-et-al-2016}. Some examples of common loss functions are \textbf{mean squared error (MSE)}, \textbf{mean absolute error (MAE)} and \textbf{mean absolute percentage error (MAPE)}. A problem with MAPE is that when the true value is 0 the function is undefined \cite{MAPE}. MSE gives a larger weight to outliers since the metric has an exponential relation to the error.

\subsection{Hidden Layers and Neuron Count}
The more hidden layers and neurons a network has the more computationally complex the learning becomes. Therefore you should not use a model that is more complex than the problem in question requires. Problems that involve large amount of input features like the \textit{ImageNet} contest has been shown to be a case where deep neural networks perform well \cite{ImageNet}. However, it has been shown that virtually any function can be approximated with two hidden layers given there are enough neurons \cite{Demuth}.

\section{Kalman Filters}
Kalman filtering is an iterative algorithm used to increase measurment accuracy by considering multiple consecutive data points observed over time. It can be used in real time as new measurements are made. Kalman filtering has been shown to be useful when predicting travel times \cite{kalmanPrediction}. It has also been used together with artificial neural networks with promising results \cite{kalmanANN}.

%The kalman filter consists of two phases, often called the \textit{predict} phase and the \textit{update} phase. CONTINUE


% The main purpose of this chapter is to make it obvious for
% the reader that the report authors have made an effort to read
% up on related research and other information of relevance for
% the research questions. It is a question of trust. Can I as a
% reader rely on what the authors are saying? If it is obvious
% that the authors know the topic area well and clearly present
% their lessons learned, it raises the perceived quality of the
% entire report.

% After having read the theory chapter it shall be obvious for
% the reader that the research questions are both well
% formulated and relevant.

% The chapter must contain theory of use for the intended
% study, both in terms of technique and method. If a final thesis
% project is about the development of a new search engine for
% a certain application domain, the theory must bring up related
% work on search algorithms and related techniques, but also
% methods for evaluating search engines, including
% performance measures such as precision, accuracy and
% recall.

% The chapter shall be structured thematically, not per author.
% A good approach to making a review of scientific literature
% is to use \emph{Google Scholar} (which also has the useful function
% \emph{Cite}). By iterating between searching for articles and reading
% abstracts to find new terms to guide further searches, it is
% fairly straight forward to locate good and relevant
% information, such as \cite{test}.

% Having found a relevant article one can use the function for
% viewing other articles that have cited this particular article,
% and also go through the article’s own reference list. Among
% these articles on can often find other interesting articles and
% thus proceed further.

% It can also be a good idea to consider which sources seem
% most relevant for the problem area at hand. Are there any
% special conference or journal that often occurs one can search
% in more detail in lists of published articles from these venues
% in particular. One can also search for the web sites of
% important authors and investigate what they have published
% in general.

% This chapter is called either \emph{Theory, Related Work}, or
% \emph{Related Research}. Check with your supervisor.


%%%%%%%%%%%%%%%%%%%%%%%%%%%%%%%%%%%%%%%%%%%%%%%%%%%%%%%%%%%%%%%%%%%%%%
%%% lorem.tex ends here

%%% Local Variables: 
%%% mode: latex
%%% TeX-master: "demothesis"
%%% End: 
