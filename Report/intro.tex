
3Arrival Time Prediction On Buses in LinköpingArrival Time Prediction On Buses in LinköpingArrival Time Prediction On Buses in Linköping%%% Intro.tex --- 
%% 
%% Filename: Intro.tex
%% Description: 
%% Author: Ola Leifler
%% Maintainer: 
%% Created: Thu Oct 14 12:54:47 2010 (CEST)
%% Version: $Id$
%% Version: 
%% Last-Updated: Thu May 19 14:12:31 2016 (+0200)
%%           By: Ola Leifler
%%     Update #: 5
%% URL: 
%% Keywords: 
%% Compatibility: 
%% 
%%%%%%%%%%%%%%%%%%%%%%%%%%%%%%%%%%%%%%%%%%%%%%%%%%%%%%%%%%%%%%%%%%%%%%
%% 
%%% Commentary: 
%% 
%% 
%% 
%%%%%%%%%%%%%%%%%%%%%%%%%%%%%%%%%%%%%%%%%%%%%%%%%%%%%%%%%%%%%%%%%%%%%%
%% 
%%% Change log:
%% 
%% 
%% RCS $Log$
%%%%%%%%%%%%%%%%%%%%%%%%%%%%%%%%%%%%%%%%%%%%%%%%%%%%%%%%%%%%%%%%%%%%%%
%% 
%%% Code:


\chapter{Introduction}
\label{cha:introduction}

In this section problem statement, goals and approach is discussed.

\section{Problem}
\label{sec:problem}

The problem at hand is to do bus time arrival prediction. 

\section{Goals}
\label{sec:aim}

\begin{enumerate}
  \item Pre-process data
  \item Predict bus time arrival with two different solutions
  \item Compare solutions
\end{enumerate}

\section{Approach}
\label{sec:research-questions}

In this report bus time arrival prediction will be studied using Gaussian proccesses and Neural networks. The theory behind the solutions will be discussed in depth in chapter \ref{cha:theory} and the methodology will be discussed in chapter \ref{cha:method}.

The solutions will be compared by their mean absolute percentage error on identical predictons (MAPE).

\section{Delimitations}
\label{sec:delimitations}

Only one bus line will be considered. Temprorary disruptions will not be considered. Spatial outliers are removed in pre-processing.

%\nocite{scigen}
%We have included Paper \ref{art:scigen}

%%%%%%%%%%%%%%%%%%%%%%%%%%%%%%%%%%%%%%%%%%%%%%%%%%%%%%%%%%%%%%%%%%%%%%
%%% Intro.tex ends here


%%% Local Variables: 
%%% mode: latex
%%% TeX-master: "demothesis"
%%% End: 
