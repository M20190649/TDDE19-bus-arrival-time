%%% lorem.tex --- 
%% 
%% Filename: lorem.tex
%% Description: 
%% Author: Ola Leifler
%% Maintainer: 
%% Created: Wed Nov 10 09:59:23 2010 (CET)
%% Version: $Id$
%% Version: 
%% Last-Updated: Wed Nov 10 09:59:47 2010 (CET)
%%           By: Ola Leifler
%%     Update #: 2
%% URL: 
%% Keywords: 
%% Compatibility: 
%% 
%%%%%%%%%%%%%%%%%%%%%%%%%%%%%%%%%%%%%%%%%%%%%%%%%%%%%%%%%%%%%%%%%%%%%%
%% 
%%% Commentary: 
%% 
%% 
%% 
%%%%%%%%%%%%%%%%%%%%%%%%%%%%%%%%%%%%%%%%%%%%%%%%%%%%%%%%%%%%%%%%%%%%%%
%% 
%%% Change log:
%% 
%% 
%% RCS $Log$
%%%%%%%%%%%%%%%%%%%%%%%%%%%%%%%%%%%%%%%%%%%%%%%%%%%%%%%%%%%%%%%%%%%%%%
%% 
%%% Code:

\chapter{Results}
\label{cha:results}
This section present results in the form of performance metrics for the investigate models. Both the NN model and the GP model performed well.

\begin{table}[H]
  \centering
  \caption{Model MAE and MAPE for test trajectories containing only the first perentage of data pionts.}
  \label{tbl:models-mae-and-mape}
  \begin{tabular}{l | l | l | l | l || l | l | l | l }
    & \multicolumn{4}{c}{MAE} & \multicolumn{4}{c}{MAPE} \\
    Model      & 20\% & 40\% & 60\% & 80\% & 20\% & 40\% & 60\% & 80\% \\
    \hline
    Mean pred. & 2 & 10 & 5 & 12  & 2 & 10 & 5 & 12 \\
    NN - M1        & 15.10 &  15.10 & 15.10 &  15.10  & 23.58\% & 26.93\% & 32.09\% & 41.91\% \\
    NN - M2        & 12.55 &  10.80 & 9.64 &  8.55  & 14.02\% & 14.66\% & 14.71\% & 15.08\% \\
    NN - M3        & 5 &  7 & 8 &  7  & 2 & 10 & 5 & 12 \\
    NN - M4        & 5 &  7 & 8 &  7  & 2 & 10 & 5 & 12 \\
    NN - M1 compr.        & 12.09 &  12.09 & 12.09 &  12.09  & 30.04\% & 35.48\% & 42.68\% & 56.11\% \\
    NN - M2 compr.       & 5.31 &  4.33 & 3.71 &  3.18  & 10.18\% & 9.32 & 9.20 & 9.36 \\
    NN - M3 compr.       & 5 &  7 & 8 &  7  & 2 & 10 & 5 & 12 \\
    NN - M4 compr.       & 5 &  7 & 8 &  7  & 2 & 10 & 5 & 12 \\
    GPR        & 5 &  7 & 8 &  7  & 2 & 10 & 5 & 12 \\
  \end{tabular}
\end{table}

\begin{table}[H]
  \centering
  \caption{MAE of models for the different training trajectories.}
  \label{tbl:model-mae-of-trajs}
  \begin{tabular}{ l | l | l | l | l | l | l | l | l | l | l | l | l }
    & \multicolumn{12}{c}{Trajectory} \\
    Model       & 1 & 2 & 3 & 4 & 5  & 5 & 6 & 3 & 4 & 5  & 5 & 6 \\
    \hline
    Mean pred.  & 6 & 4 & 0 & 2 & 10 & 5 & 12 & 3 & 4 & 5  & 5 & 6 \\
    NN1         & 6 & 3 & 0 & 3 &  8 & 9 &  9 & 3 & 4 & 5  & 5 & 6 \\
    GPR         & 6 & 2 & 1 & 3 &  7 & 8 &  7 & 3 & 4 & 5  & 5 & 6 \\
  \end{tabular}
\end{table}

\begin{table}[H]
  \centering
  \caption{MAPE of models for the different training trajectories.}
  \label{fig:model-mape-of-trajs}
  \begin{tabular}{l | l | l | l | l | l | l | l | l | l | l | l | l }
    & \multicolumn{12}{c}{Trajectory} \\
    Model        & 1 & 2 & 3 & 4 & 5  & 5 & 6 & 3 & 4 & 5  & 5 & 6 \\
    \hline
    Mean pred.  & 6 & 4 & 0 & 2 & 10 & 5 & 12 & 3 & 4 & 5  & 5 & 6 \\
    NN1         & 6 & 3 & 0 & 3 &  8 & 9 &  9 & 3 & 4 & 5  & 5 & 6 \\
    GPR         & 6 & 2 & 1 & 3 &  7 & 8 &  7 & 3 & 4 & 5  & 5 & 6 \\
  \end{tabular}
\end{table}

% This chapter presents the results. Note that the results are presented
% factually, striving for objectivity as far as possible.  The results
% shall not be analyzed, discussed or evaluated.  This is left for the
% discussion chapter.

% In case the method chapter has been divided into subheadings such as
% pre-study, implementation and evaluation, the result chapter should
% have the same sub-headings. This gives a clear structure and makes the
% chapter easier to write.

% In case results are presented from a process (e.g. an implementation
% process), the main decisions made during the process must be clearly
% presented and justified. Normally, alternative attempts, etc, have
% already been described in the theory chapter, making it possible to
% refer to it as part of the justification.

%%%%%%%%%%%%%%%%%%%%%%%%%%%%%%%%%%%%%%%%%%%%%%%%%%%%%%%%%%%%%%%%%%%%%%
%%% lorem.tex ends here

%%% Local Variables: 
%%% mode: latex
%%% TeX-master: "demothesis"
%%% End: 
