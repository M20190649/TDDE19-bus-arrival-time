%%% lorem.tex --- 
%% 
%% Filename: lorem.tex
%% Description: 
%% Author: Ola Leifler
%% Maintainer: 
%% Created: Wed Nov 10 09:59:23 2010 (CET)
%% Version: $Id$
%% Version: 
%% Last-Updated: Wed Nov 10 09:59:47 2010 (CET)
%%           By: Ola Leifler
%%     Update #: 2
%% URL: 
%% Keywords: 
%% Compatibility: 
%% 
%%%%%%%%%%%%%%%%%%%%%%%%%%%%%%%%%%%%%%%%%%%%%%%%%%%%%%%%%%%%%%%%%%%%%%
%% 
%%% Commentary: 
%% 
%% 
%% 
%%%%%%%%%%%%%%%%%%%%%%%%%%%%%%%%%%%%%%%%%%%%%%%%%%%%%%%%%%%%%%%%%%%%%%
%% 
%%% Change log:
%% 
%% 
%% RCS $Log$
%%%%%%%%%%%%%%%%%%%%%%%%%%%%%%%%%%%%%%%%%%%%%%%%%%%%%%%%%%%%%%%%%%%%%%
%% 
%%% Code:

%Hack to make the captions look nice:
\captionsetup{width=.75\textwidth}


\chapter{Results}
\label{cha:results}
This section present results in the form of performance metrics for the investigate models. Both the NN model and the GP model performed well.

\section{Bus line 3}
The tables below presents the results from model predictions on journeys of bus line 3.
\begin{table}[H]
  \centering
  \caption{Model MAE and MAPE for test trajectories in bus line 3 containing only the first percentage of data points.}
  \label{tbl:models-mae-and-mape-203}
  \begin{tabular}{l | l | l | l | l || l | l | l | l }
    & \multicolumn{4}{c}{MAE} & \multicolumn{4}{c}{MAPE} \\
    Model      & 20\% & 40\% & 60\% & 80\% & 20\% & 40\% & 60\% & 80\% \\
    \hline
    Baseline & 14.9 & 14.9 & 14.9 & 14.9  & 27.7 & 37.0 & 56.3 & 119.9 \\
    NN1        & 4 &  8 & 9 &  9  & 2 & 10 & 5 & 12 \\
    GPR        & 5 &  7 & 8 &  7  & 2 & 10 & 5 & 12 \\
  \end{tabular}
\end{table}

\begin{table}[H]
  \centering
  \caption{MAE of models for the different segments in bus line 3.}
  \label{tbl:model-mae-of-segs-203}
  \begin{tabular}{ l | l | l | l | l | l | l | l | l | l | l | l }
    & \multicolumn{11}{c}{Segment} \\
    Model       & 1 & 2 & 3 & 4 & 5 & 6 & 7 & 8 & 9 & 10 & 11 \\
    \hline
    Baseline  & 27.0 & 11.4 & 10.3 & 13.0 & 10.6 & 12.0 & 14.9 & 12.2 & 19.3 & 16.1  & 16.5 \\
    NN1         & 6 & 3 & 0 & 3 &  8 & 9 &  9 & 3 & 4 & 5  & 5 \\
    GPR         & 6 & 2 & 1 & 3 &  7 & 8 &  7 & 3 & 4 & 5  & 5 \\
  \end{tabular}
\end{table}

\begin{table}[H]
  \centering
  \caption{MAPE of models for the different segments in bus line 3.}
  \label{fig:model-mape-of-segs-203}
\begin{tabular}{ l | l | l | l | l | l | l | l | l | l | l | l }
	& \multicolumn{11}{c}{Segment} \\
	Model       & 1 & 2 & 3 & 4 & 5 & 6 & 7 & 8 & 9 & 10 & 11 \\
	\hline
	Baseline  & 43.0 & 63.3 & 48.8 & 111.7 & 44.0 & 57.6 & 35.8 & 98.0 & 29.9 & 59.4  & 71.0 \\
	NN1         & 6 & 3 & 0 & 3 &  8 & 9 &  9 & 3 & 4 & 5  & 5 \\
	GPR         & 6 & 2 & 1 & 3 &  7 & 8 &  7 & 3 & 4 & 5  & 5 \\
\end{tabular}
\end{table}

\section{Bus line 11}
The tables below presents the results from model predictions on journeys of bus line 11. The reason for only using the ANN and baseline models for this bus line is explained in the discussion \ref{cha:discussion}.
\begin{table}[H]
	\centering
	\caption{Model MAE and MAPE for test trajectories in bus line 11. containing only the first percentage of data points}
	\label{tbl:models-mae-and-mape-211}
	\begin{tabular}{l | l | l | l | l || l | l | l | l }
		& \multicolumn{3}{c}{MAE} & \multicolumn{4}{c}{MAPE} \\
		Model      & 20\% & 40\% & 60\% & 80\% & 20\% & 40\% & 60\% & 80\% \\
		\hline
		Baseline & 16.6 & 16.6 & 16.6 & 16.6  & 25.6 & 34.1 & 51.7 & 109.0 \\
		NN1        & 4 &  8 & 9 &  9  & 2 & 10 & 5 & 12 \\
	\end{tabular}
\end{table}

\begin{table}[H]
	\centering
	\caption{MAE of models for the different segments in bus line 11.}
	\label{tbl:model-mae-of-segs-211}
	\begin{tabular}{ l | l | l | l | l | l | l | l | l | l | l | l | l}
		& \multicolumn{12}{c}{Segment} \\
		Model       & 1 & 2 & 3 & 4 & 5 & 6 & 7 & 8 & 9 & 10 & 11 & 12 \\
		\hline
		Baseline  & 38.4 & 26.1 & 29.3 & 8.1 & 10.7 & 9.2 & 11.6 & 18.5 & 9.1 & 9.1  & 12.0 & 16.6 \\
		NN1         & 6 & 3 & 0 & 3 &  8 & 9 &  9 & 3 & 4 & 5  & 5 & 8\\
	\end{tabular}
\end{table}

\begin{table}[H]
	\centering
	\caption{MAPE of models for the different segments in bus line 11.}
	\label{fig:model-mape-of-segs-211}
	\begin{tabular}{ l | l | l | l | l | l | l | l | l | l | l | l | l}
		& \multicolumn{12}{c}{Segment} \\
		Model       & 1 & 2 & 3 & 4 & 5 & 6 & 7 & 8 & 9 & 10 & 11 & 12 \\
		\hline
		Baseline  & 48.6 & 72.6 & 46.6 & 61.8 & 71.0 & 50.6 & 39.1 & 44.7 & 71.3 & 49.3  & 42.6 & 63.3 \\
		NN1         & 6 & 3 & 0 & 3 &  8 & 9 &  9 & 3 & 4 & 5  & 5 & 8\\
	\end{tabular}
\end{table}

% This chapter presents the results. Note that the results are presented
% factually, striving for objectivity as far as possible.  The results
% shall not be analyzed, discussed or evaluated.  This is left for the
% discussion chapter.

% In case the method chapter has been divided into subheadings such as
% pre-study, implementation and evaluation, the result chapter should
% have the same sub-headings. This gives a clear structure and makes the
% chapter easier to write.

% In case results are presented from a process (e.g. an implementation
% process), the main decisions made during the process must be clearly
% presented and justified. Normally, alternative attempts, etc, have
% already been described in the theory chapter, making it possible to
% refer to it as part of the justification.

%%%%%%%%%%%%%%%%%%%%%%%%%%%%%%%%%%%%%%%%%%%%%%%%%%%%%%%%%%%%%%%%%%%%%%
%%% lorem.tex ends here

%%% Local Variables: 
%%% mode: latex
%%% TeX-master: "demothesis"
%%% End: 
