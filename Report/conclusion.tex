%%% lorem.tex --- 
%% 
%% Filename: lorem.tex
%% Description: 
%% Author: Ola Leifler
%% Maintainer: 
%% Created: Wed Nov 10 09:59:23 2010 (CET)
%% Version: $Id$
%% Version: 
%% Last-Updated: Wed Nov 10 09:59:47 2010 (CET)
%%           By: Ola Leifler
%%     Update #: 2
%% URL: 
%% Keywords: 
%% Compatibility: 
%% 
%%%%%%%%%%%%%%%%%%%%%%%%%%%%%%%%%%%%%%%%%%%%%%%%%%%%%%%%%%%%%%%%%%%%%%
%% 
%%% Commentary: 
%% 
%% 
%% 
%%%%%%%%%%%%%%%%%%%%%%%%%%%%%%%%%%%%%%%%%%%%%%%%%%%%%%%%%%%%%%%%%%%%%%
%% 
%%% Change log:
%% 
%% 
%% RCS $Log$
%%%%%%%%%%%%%%%%%%%%%%%%%%%%%%%%%%%%%%%%%%%%%%%%%%%%%%%%%%%%%%%%%%%%%%
%% 
%%% Code:

\chapter{Conclusion}
\label{cha:conclusion}

We have shown that we could make good predictions on bus arrival times. blabla

\subsection{Future Work}
Having the time schedule of the buses available would allow for some interesting new approaches. At any given station the current delay could be fed into the neural network. Since bus driver drive faster and dwell less when they are late, having this as an input could improve prediction accuracy. Furthermore a very simple baseline model could be created using the time schedule. By simple using the offset of the last bus it should be possible to make acceptable predictions, as two subsequent bus often have similar delays due to traffic conditions.

The recurrent neural network model (ANN model 4) could possibly perform better if a longer sequence of data points were used. Due to hardware constraints the model in this work uses sequences of 20 data points as input, more than that required additional RAM. Doing this would either require more capable hardware or some exploration of how to split the training up into multiple sessions where the data is divided into smaller chunks.
%%%%%%%%%%%%%%%%%%%%%%%%%%%%%%%%%%%%%%%%%%%%%%%%%%%%%%%%%%%%%%%%%%%%%%
%%% lorem.tex ends here

%%% Local Variables: 
%%% mode: latex
%%% TeX-master: "demothesis"
%%% End: 
